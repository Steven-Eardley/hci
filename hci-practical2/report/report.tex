\documentclass[a4paper,11pt,oneside]{article}

\usepackage{graphicx}

\begin{document}

\title{HCI Coursework 2}
\author{s0934142 \and s0901522}
\maketitle

\section{Cooperative Evaluation}
We had two subjects use both the software we were evaluating and our software from part one.  We gave them a set of tasks (Appendix~\ref{sec:tasks}) designed test all the functionality of both systems.  We then timed them performing the tasks, made observations on how they were interacting with the systems and then had the complete a survey for both pieces of software to gauge their reactions.

JUSTIFICATION OF TASKS

Both subjects were naive users, and we tried let them work out what to do without prompting, although we did prompt if they became very stuck.

Software 1 is the software given to us to evaluate and software 2 is our software we developed in part one.

\subsection{Subject A}

\subsubsection{Software 1}
It took the subject 8min51s to complete the set of tasks. 

Our observations noticed that throughout the tasks the user was asking questions on what to do and how to do it, they had to refer to the help guide multiple times, including for the same task: how to go into edit mode.  The subjet also started to display rage, at one point swearing at the computer.  They attempted to apply external knowledge in the form of keyboard shortcuts.  They were also appeared frustrated when pressing buttons that had no feedback indicating success or failure.  When the subject was required to move the shape to a new lamppost they tried to move the whole shape in one, not move it point by point.

SURVEY RESULTS

\subsubsection{Software 2}
It took the subject 5min20s to complete the set of tasks.

Our observations noticed that the icons on the button seemed to help the user work out what to do.  The subject coped fairly well with this software until it came to editing and deleting.  We had to explain to the subject that to move a marking it had to be deleted and redrawn.  Attempting to edit again induced a rage with the subject swearing at the computer.  The user had to be prompted to click in the label box on the right hand side to select object ready to be marked or deleted.  They did only need to be prompted on this the first time.

SURVEY RESULTS

\subsection{Subject B}
\subsubsection{Software 1}
It took the subject 8min29s to complete the set of tasks.

Our observations noticed that straight away the subject struggled with opening an image, this software displayed all files, not just image files which confused the subject.  Also when the subject needed to use the help guide it was described as "too wordy".  The subject also struggled with moving the markings, attempting to move without being in the correct mode.  Switching modes had to be prompted to them, and they had to be prompted every time the mode needed to be switched.

SURVEY RESULTS



\subsubsection{Software 2}
It took the subject 6min56s to complete the set of tasks.

\subsection{Findings}
Both subjects said both pieces of software made them feel stupid.

One potential issue with our experiment is that by having the users use two pieces of software is that when they use the second piece of software they are not necessarily naive users, they may have learned from the first software.  This could be the cause of the shorted times both users had on software two.  One way we could have mitigated this is by having each user use the software in a different order, unfortunately we did not do this.

DYSLEXIC AND COLOUR BLIND

\section{Evaluation and Comparison}

\newpage
\section{Appendix}
\appendix

\section{User Tasks}
\label{sec:tasks}

\begin{enumerate}
\item Open an image
\item Mark an object
\item Add the label
\item Edit the label
\item Mark another object
\item Save
\item Open another image
\item Mark a lamppost
\item Move these markings to a different lamppost
\item Open up the original image
\item Load the data
\item Delete an object
\item Close the program
\end{enumerate}

We instructed the user to use a directory which contained two images, both with multiple lampposts in.

\end{document}
