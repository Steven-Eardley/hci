\documentclass[a4paper,11pt,oneside]{article}

\usepackage{graphicx}
\usepackage{multirow}

\begin{document}

\title{HCI Coursework 2}
\author{s0934142 \and s0901522}
\maketitle

\section{Cooperative Evaluation}
We had two subjects use both the software we were evaluating and our software from part one.  We gave them a set of tasks (Appendix~\ref{sec:tasks}) designed test all the functionality of both systems.  We then timed them performing the tasks, made observations on how they were interacting with the systems and then had the complete a survey for both pieces of software to gauge their reactions.

We chose this set of tasks as it covered all the visible functionality of the software we were investigating as well as covering our software (apart from launching the help guide).  And the tasks seemed to us like a reasonable approximation of a user using the software.

Both subjects were non-informatics naive users, and we tried let them work out what to do without prompting, although we did prompt if they became very stuck.

Software 1 is the software given to us to evaluate and software 2 is our software we developed in part one.

\subsection{Subject A}

\subsubsection{Software 1}
It took the subject 8min51s to complete the set of tasks. 

Our observations noticed that throughout the tasks the user was asking questions on what to do and how to do it, they had to refer to the help guide multiple times, including for the same task: how to go into edit mode.  The subjet also started to display rage, at one point swearing at the computer.  They attempted to apply external knowledge in the form of keyboard shortcuts.  They were also appeared frustrated when pressing buttons that had no feedback indicating success or failure.  When the subject was required to move the shape to a new lamppost they tried to move the whole shape in one, not move it point by point.

The survey (Appendix~\ref{sec:survey}, Table~\ref{tab:a1}) taken at the end indicates in general that they thought the software was good, the only area that really needs improvement would be the appearance.

\subsubsection{Software 2}
It took the subject 5min20s to complete the set of tasks.

Our observations noticed that the icons on the button seemed to help the user work out what to do, and the textual tooltips shown when the mouse was hovered over a button aided the user to confirm the use until the function was remembered.  The subject coped fairly well with this software until it came to editing and deleting.  We had to explain to the subject that to move a marking it had to be deleted and redrawn.  Attempting to edit again induced a rage with the subject swearing at the computer.  The user had to be prompted to click in the label box on the right hand side to select object ready to be marked or deleted.  They did only need to be prompted on this the first time. The subject did not find the right-click shortcut to undo drawing, which caused some confusion throughout the tasks.

The survey (Appendix~\ref{sec:survey}, Table~\ref{tab:a2}) taken at the end indicates that in general they thought the software was good.  There is only one area that needs improvment and that is deleting a marked object.  It is interesting that they gave deleting and editing different scores as they require the same procedure.

\subsection{Subject B}
\subsubsection{Software 1}
It took the subject 8min29s to complete the set of tasks.

Our observations noticed that straight away the subject struggled with opening an image, this software displayed all files, not just image files which confused the subject.  Also when the subject needed to use the help guide it was described as "too wordy".  The subject also struggled with moving the markings, attempting to move without being in the correct mode.  Switching modes had to be prompted to them, and they had to be prompted every time the mode needed to be switched.

The survey (Appendix~\ref{sec:survey}, Table~\ref{tab:b1}) taken at the end indicates mixed feelings about this software.  The core functionality of marking objects is marked as not easy, and the subject did not like the appearence of the software and overall had a negative experience.


\subsubsection{Software 2}
It took the subject 6min56s to complete the set of tasks.

Our observations noticed that again using the listbox needed to be more prominent as we had to prompt the subject to use that to allow editing and deleting.  Again we had to explain that moving required deleting and redrawing.  The subject did discover one of our keyboard shortcuts: ctrl-z for undo.  They then proceeded to use this throughout the task.

The survey (Appendix~\ref{sec:survey}, Table~\ref{tab:b2}) taken at the end indicates the subject had a more positive experience than with the first piece of software: there were no negative marks.  Overall the subject indicating a strong positive experience.  Interestingly there is a big difference in how easy this subject found deleting objects than subject A, this could be because editing is covered first and the two procedures are similar so when deleting was required they already knew the prerequisites for this.

\subsection{Findings}
Both subjects said both pieces of software made them feel stupid, despite the high ratings they awarded.  Which inidicates that both pieces of software failed at being user friendly.

One potential issue with our experiment is that by having the users use two pieces of software is that when they use the second piece of software they are not necessarily naive users, they may have learned from the first software.  This could be the cause of the shorted times both users had on software two.  One way we could have mitigated this is by having each user use the software in a different order, unfortunately we did not do this.

One of the subjects was dyslexic and colour blind and was dissapointed that neither software had any provision for either of these.  This may have had a performance on the subject's performance on the tasks.

For software 1 anything relating to editing caused a lot of problems for the user, this we believe is due to the unintuitive 'toggle mode' feature.  For software 2 it was deleting and loading data that were identified as causing problems.

A more detailed analysis of the areas for improvements in software 2, including survey results and observations, shows that: the documentation, toggling of modes, provide feedback from interactions, moving the shape as a whole not point by point, filtering of files and a general improvement in aesthetic are the todo points for the software.

Whilst we cannot be 100\% sure, due to the way the experiment was run, it was in software 2 that the tasks were completed faster.  Our observations of the subjects noticed less problems with software 2 than software 1.  The survey results themselves also indicate a slight preference for software 2 with both users having a more positive experience with software 2.  However as both pieces of software made them feel stupid we know that both need serious improvement.

\section{Evaluation and Comparison}

The software we were given to evaluate does not adhere particularly well to Shneiderman's 8 Golden Rules (Appendix~\ref{sec:s_rules}), with breaches of rules 3, 4, 6 and 8. For Rule 3, feedback is a problem because when the mode has been changed via the 'Toggle mode' button there is no indication of what each mode is for. The user is expected to know which mode they are in by the options available - the 'greying out' of the 'Finish current label' button is helpful but does not redeem this enough. The breach of Rule 4 occurs when we first open the application; the user is immediately confronted by a standard open file dialogue, but no indication that the tagger itself is running. This can be a little disconcerting. Additionally, whenever the load tags button is clicked the user is prompted that current changes will be lost, even if there aren't any unsaved changes in the space, or indeed any polygons at all. This needlessly interrupts user flow. The program also doesn't include an 'undo' functionality, so breaks Rule 6. Because of this, to cancel a point drawn in error, an entire polygon must be created, labelled and deleted. Rule 8 fails to be met because changing mode adds needless cognitive load: remembering which mode the program is in, and the sequence of actions to change it. As discussed above, it was found to cause test subjects some difficulty.

The evaluation program does conform well with rules 2 and 7: shortcuts exist in the standard menu fashion, i.e. pressing the 'alt' key with the underlined letter performs that button click. This is an effective way of speeding up the process.

To improve compliance with Shneiderman's Golden Rules, the 'Toggle Mode' button could be renamed to show the current mode, e.g. 'Switch to edit mode' when drawing and 'Return to draw mode' when editing. Likewise, the 

\newpage
\section{Appendix}
\appendix

\section{User Tasks}
\label{sec:tasks}

\begin{enumerate}
\item Open an image
\item Mark an object
\item Add the label
\item Edit the label
\item Mark another object
\item Save
\item Open another image
\item Mark a lamppost
\item Move these markings to a different lamppost
\item Open up the original image
\item Load the data
\item Delete an object
\item Close the program
\end{enumerate}

We instructed the user to use a directory which contained two images, both with multiple lampposts in so that the tasks could be acheived no matter which image was chosen first by the subject.

\section{Shneiderman’s 8 Golden Rules (1987)}
\label{sec:s_rules}
\begin{enumerate}
\item Strive for consistency
\item Enable frequent users to use shortcuts
\item Offer informative feedback
\item Design dialogs to yield closure
\item Offer error prevention and simple error handling
\item Permit easy reversal of actions
\item Support internal locus of control
\item Reduce short-term memory load 
\end{enumerate}

\section{Nielsen’s 10 Usability Heuristics (1994)}
\label{sec:n_rules}
\begin{enumerate}
\item Visibility of system status
\item Match between system and the real world
\item User control and freedom
\item Consistency and standards
\item Help users recognize, diagnose and recover from errors
\item Error prevention
\item Recognition rather than recall
\item Flexibility and efficiency of use
\item Aesthetic and minimalist design
\item Help and documentation
\end{enumerate}

\section{Survey Results}
\label{sec:survey}

\begin{table}[h]

\centering
\caption{Subject A, Software 1}

\vspace{10px}
\begin{tabular}{ p{7cm} | c | c | c | c | c | c | c | }
    \cline{2-8}
     & \multicolumn{3}{|l}{No} & \multicolumn{1}{c}{ } & \multicolumn{3}{r|}{Yes} \\
    \cline{2-8}
      & 1 & 2 & 3 & 4 & 5 & 6 & 7 \\
    \hline
    \multicolumn{1}{|l|}{Was it easy to mark an object?} & & & & & & x & \\
    \hline
    \multicolumn{1}{|l|}{Was it easy to edit a label?} & & & & & x & & \\
    \hline
    \multicolumn{1}{|l|}{Was it easy to delete an object?} & & & & & & x & \\
    \hline
    \multicolumn{1}{|l|}{Was it easy to save your work?} & & & & & & x & \\
    \hline
    \multicolumn{1}{|l|}{Was it easy to load previous work?} & & & & & & x & \\
    \hline
    \multicolumn{1}{|l|}{Was it easy to open a new image?} & & & & & & x & \\
    \hline
    \multicolumn{1}{|l|}{Did you find the appearance appealing?} & & & & x & & & \\
    \hline
    \multicolumn{1}{|l|}{Did you have a positive experience?} & & & & & x & & \\
    \hline
\end{tabular} 
\label{tab:a1}
\end{table}

\begin{table}[h]
\centering
\caption{Subject A, Software 2}

\vspace{10px}
\begin{tabular}{ p{7cm} | c | c | c | c | c | c | c | }
    \cline{2-8}
     & \multicolumn{3}{|l}{No} & \multicolumn{1}{c}{ } & \multicolumn{3}{r|}{Yes} \\
    \cline{2-8}
      & 1 & 2 & 3 & 4 & 5 & 6 & 7 \\
    \hline
    \multicolumn{1}{|l|}{Was it easy to mark an object?} & & & & & &  & x \\
    \hline
    \multicolumn{1}{|l|}{Was it easy to edit a label?} & & & & &  & x & \\
    \hline
    \multicolumn{1}{|l|}{Was it easy to delete an object?} & & & x & & &  & \\
    \hline
    \multicolumn{1}{|l|}{Was it easy to save your work?} & & & & & & x & \\
    \hline
    \multicolumn{1}{|l|}{Was it easy to load previous work?} & & & & & & x & \\
    \hline
    \multicolumn{1}{|l|}{Was it easy to open a new image?} & & & & & & x & \\
    \hline
    \multicolumn{1}{|l|}{Did you find the appearance appealing?} & & & &  & & x & \\
    \hline
    \multicolumn{1}{|l|}{Did you have a positive experience?} & & & & &  & x & \\
    \hline
\end{tabular}
\label{tab:a2}
\end{table}

\begin{table}[h]

\centering
\caption{Subject B, Software 1}

\vspace{10px}
\begin{tabular}{ p{7cm} | c | c | c | c | c | c | c | }
    \cline{2-8}
     & \multicolumn{3}{|l}{No} & \multicolumn{1}{c}{ } & \multicolumn{3}{r|}{Yes} \\
    \cline{2-8}
      & 1 & 2 & 3 & 4 & 5 & 6 & 7 \\
    \hline
    \multicolumn{1}{|l|}{Was it easy to mark an object?} & & & & x & &  &  \\
    \hline
    \multicolumn{1}{|l|}{Was it easy to edit a label?} & & & x & &  &  & \\
    \hline
    \multicolumn{1}{|l|}{Was it easy to delete an object?} & & &  & & & x & \\
    \hline
    \multicolumn{1}{|l|}{Was it easy to save your work?} & & & & & x &  & \\
    \hline
    \multicolumn{1}{|l|}{Was it easy to load previous work?} & & & & x & &  & \\
    \hline
    \multicolumn{1}{|l|}{Was it easy to open a new image?} & & & & & & x & \\
    \hline
    \multicolumn{1}{|l|}{Did you find the appearance appealing?} & & & x &  & &  & \\
    \hline
    \multicolumn{1}{|l|}{Did you have a positive experience?} & & & x & &  &  & \\
    \hline
\end{tabular} 
\label{tab:b1}
\end{table}

\begin{table}[h]

\centering
\caption{Subject B, Software 2}

\vspace{10px}
\begin{tabular}{ p{7cm} | c | c | c | c | c | c | c | }
    \cline{2-8}
     & \multicolumn{3}{|l}{No} & \multicolumn{1}{c}{ } & \multicolumn{3}{r|}{Yes} \\
    \cline{2-8}
      & 1 & 2 & 3 & 4 & 5 & 6 & 7 \\
    \hline
    \multicolumn{1}{|l|}{Was it easy to mark an object?} & & & & & & x &  \\
    \hline
    \multicolumn{1}{|l|}{Was it easy to edit a label?} & & & & &  & x & \\
    \hline
    \multicolumn{1}{|l|}{Was it easy to delete an object?} & & &  & & &  & x \\
    \hline
    \multicolumn{1}{|l|}{Was it easy to save your work?} & & & & & x &  & \\
    \hline
    \multicolumn{1}{|l|}{Was it easy to load previous work?} & & & & x & &  & \\
    \hline
    \multicolumn{1}{|l|}{Was it easy to open a new image?} & & & & & x &  & \\
    \hline
    \multicolumn{1}{|l|}{Did you find the appearance appealing?} & & & &  & x &  & \\
    \hline
    \multicolumn{1}{|l|}{Did you have a positive experience?} & & & & &  & x & \\
    \hline
\end{tabular} 
\label{tab:b2}
\end{table}


\end{document}
